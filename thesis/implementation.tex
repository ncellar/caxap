\chapter{Implementation \& Internals}

In this section, we look at what caxap has under the hood. We start with an
overview of the whole implementation, and then linger on a few topics that were
tricky to implement and/or design.

This chapter assumes you have read - and hopefully understood - the user manual
(chapter \ref{manual}).

%%%%%%%%%%%%%%%%%%%%%%%%%%%%%%%%%%%%%%%%%%%%%%%%%%%%%%%%%%%%%%%%%%%%%%%%%%%%%%%%
\section{Overview of the Architecture}

In this section, we briefly describe how caxap is implemented. We do so by
giving a cursory overview of each package in the caxap source tree.

%%%%%%%%%%%%%%%%%%%%%%%%%%%%%%%%%%%%%%%%
\subsection{driver}

The \texttt{EntryPoint} class is - unsurprisingly - caxap's entry point. It
processes command line options and makes them available to other parts of the
program via the singleton \texttt{Config} class. It then finds all source files
and uses the \texttt{DependencyResolver} class to order them. Finally, it passes
control to the class \texttt{CompilationDriver} which will proceed to compile
and/or expand the files in the given order.

The \texttt{CompilationDriver} has ties to the \texttt{compiler.java} package,
in order to compile expanded sources and load them in caxap's JVM.

The class \texttt{SourceFile} models each found source file. Each instance of
\texttt{SourceFile} uniquely identifies a source file. All instances of this
class must be obtained via the \texttt{SourceRepository} class, which enforces
the one-to-one mapping between \texttt{SourceFile} instances and files.

Each \texttt{SourceFile} maintains a list of the macros it defines, as well as a
list of dependencies under the form of a \texttt{Requires} object. Each instance
also has its own \texttt{SourceParseManager} which calls into the
\texttt{parser} package to parse either the file prelude or the whole file. The
file prelude consists of the package declaration and of the
\texttt{import}/\texttt{require} statements. Parsing the prelude is actually
delegated to the \texttt{RequiresParser} class.

The \texttt{Context} class makes some data globally accessible to avoid passing
it over the call stack, or to let callbacks manipulate it.

The \texttt{Hints} class is an early attempt to provide better diagnostics by
providing some context that can be used in error messages.

%%%%%%%%%%%%%%%%%%%%%%%%%%%%%%%%%%%%%%%%
\subsection{files}

Classes in this package model source files on the disk and how they relate with
the package structure. The \texttt{Require} class models as ingle
\texttt{require} statement. These classes are mostly used in the \texttt{driver}
package.

%%%%%%%%%%%%%%%%%%%%%%%%%%%%%%%%%%%%%%%%
\subsection{grammar}
\label{implem_grammar}

\texttt{Expression} is the most important class in this package. It models a PEG
parsing expression. There is one nested class per type of expression.

The base grammar for Java is defined in the \texttt{grammar.java} package, using
a domain specific language (DSL) defined in \texttt{GrammarDSL}.

In PEG, expressions form graphs, not trees, because of recursion. Building a
graph in one step is tricky in Java, since you can't use a uninitialized
variable in its definition. For instance you can't write:

\begin{lstlisting}
Expression arguments = rule_seq(identifier, opt(comma, arguments));
\end{lstlisting}

You have to use an explicit reference instead:

\begin{lstlisting}
Expression arguments = rule_seq(identifier, opt(comma, ref("arguments"));
\end{lstlisting}

Such references then have to be resolved. This is, among other things, the role
of the \texttt{ExpressionTreeCleaner} class. It also computes a textual
representation for the expression, using the notation introduced in sections
\ref{peg_formalism} and \ref{extended_notation}. Finally, it compacts the graph
by merging identical expressions. This serves to make memoization more
efficient.

The \texttt{Grammar} class represents a PEG grammar. Its constructor takes a
class as parameter (such as those in the \texttt{grammar.java} package), and
uses reflection to get the rule names from the field names. A grammar keeps a
map from rule names to the corresponding \texttt{Expression}.

The \texttt{ExpressionVisitor} interface is simply a programming trick that
allows the implementing classes to specialize some behavior depending on the
run-time type of an object with static type \texttt{Expression}. Calling
\texttt{expression.accept(visitor)} will call the appropriate \texttt{visit()}
method depending on the expression type.

The \texttt{MatchCallbacks} class defines a number of methods called on a
\texttt{Match} object at different time during its lifetime: when it is
constructed during the parse, before and after macro expansion.

Callbacks are used to implement some important features, such as compiling and
registering macro definitions. This means that that if a macro definitions is
parsed, it must make it to the final match tree, because there is no way to
unregister a macro if the parser backtracks over the macro definition.

Albeit the feature is not documented, it is possible for users to define their
own callbacks. In the future, I hope to add some new syntax and functionalities
to make the feature more transparent.

The \texttt{Expression} and \texttt{ExpressionVisitor} classes are heavily
inspired by similar classes written by Roman R. Redziejowski for the Mouse PEG
parser generator. \cite{redziejowski2009}

%%%%%%%%%%%%%%%%%%%%%%%%%%%%%%%%%%%%%%%%
\subsection{source}

The \texttt{Source} interface describes a container of source code. The
\texttt{SourceStream} class associates a \texttt{Source} object with position
within that source. Each instance of our parser class is associated to a
\texttt{Source} object, and uses a \texttt{SourceStream} to keep track of its
position in the source.

The package also contains a few implementation of the \texttt{Source}
interface. \texttt{SourceFileText} represents a source file whose contents have
been read into memory. \texttt{SourceString} represents code contained in a Java
string. \texttt{SourceComposed} represents code pieced together to support the
\texttt{MatchCreator.new_match()} methods described in section
\ref{match_class}.

The \texttt{Source} interface and most of the implementing classes were
originally written by Roman R. Redziejowski for the Mouse PEG parser
\cite{redziejowski2009}. I only made minor adaptations to his code.

%%%%%%%%%%%%%%%%%%%%%%%%%%%%%%%%%%%%%%%%
\subsection{grammar.java}

The package comprises five special files starting with the
\texttt{_<uppercase-leter>_} pattern. The fields in those classes represent the
rules of the base Java grammar. The grammar was so large that it was indeed
preferable to split it across multiple files. The lexical ordering hints at the
inheritance hierarchy: E extends D extends C ...

\texttt{JavaGrammar} extends the E class. The base Java grammar is obtained by
passing \texttt{JavaGrammar.class} to the constructor of the
\texttt{grammar.Grammar} class.

The remaining classes are \texttt{grammar.MatchCallbacks} implementations.

\texttt{CallbacksExpression} builds \texttt{Expression} objects from a macro's
syntactic specification. I hesitated to implement the features as a form of
bootstrapped macro instead.

\texttt{CallbacksMacroDefinition} handles the compilation and registration of a
macro right after its definition has been parsed.

\texttt{CallbacksPrelude} expands \texttt{require} statements into
\texttt{import} statements, or swallows them, in the case of macro requires. In
practice, it does not really expand anything but rather replaces all
\texttt{import} and \texttt{require} statements with a list of \texttt{import}
statement that was computed when the file prelude was first parsed.

%%%%%%%%%%%%%%%%%%%%%%%%%%%%%%%%%%%%%%%%
\subsection{parser}

This package holds the implementation of our PEG parser.

The \texttt{Match} class is described in details in sections \ref{match_class}
and \ref{match_api} of the user manual.

The core of the parser is in the \texttt{Matcher} class. Its constructor takes
an instance of \texttt{source.Source} and wraps it in a
\texttt{source.SourceStream}, which can then be used to find a \texttt{Match}
for a given \texttt{Expression} at the current position in the source.

We tried to keep the code clean, but there is nothing particularly fancy about
the implementation: we use the memoized depth-first walk of the expression graph
which is typical of packrat parsers (see section \ref{packrat_parsing} or
\cite{ford2002}). Memoization is handled by the \texttt{Memo} class. For each
rule it has to parse, the parser calls the method \texttt{Memo\#get(int position,
  Expression expr)}. That method either returns a memoized \texttt{ParseData}
object; either calls back into \texttt{Matcher} to get such an object, which is
then memoized and returned.

\texttt{Memo} is actually an interface. We supply three implementations of it,
that each represents a different memoization strategy or implementation. Those
are \texttt{NestedMemo}, \texttt{FlatMemo} and \texttt{LimitedMemo}. They are
described in section \ref{implem_perf}.

\texttt{Matcher} implements \texttt{grammar.ExpressionVisitor} in order to parse
each expression according to its specificities.

Each time the matcher attempts to find an expression match at a given position
in the source file, a \texttt{ParseData} object is created if none have been
memoized. This object records whether the match was successful. If it was, it
holds the corresponding \texttt{Match} object. In all cases, it also holds a
\texttt{ParseErrors} object containing the encountered parse errors. The error
reporting strategy is described in section \ref{error_reporting_manual}.

We need the \texttt{ParseData} class in addition to the \texttt{Match} class,
because \texttt{Match} is not designed to hold temporary information needed
while parsing. In particular, in case of match failure, we need to record the
failure and its cause. This is important, because failing to match an expression
can be explained by failing to match one of its sub-expressions.

Regardless of the result of an expression match, its \texttt{ParseData} is
merged into the \texttt{ParseData} of its parent: this registers a sub-match in
case of success, and merges error information.

%%%%%%%%%%%%%%%%%%%%%%%%%%%%%%%%%%%%%%%%
\subsection{trees}

The \texttt{MatchSpec} class is described in section \ref{matchspec_manual}.

The \texttt{MatchFinder} class is the implementation backbone of the
match-finding methods described in section \ref{finding_submatches}.

\texttt{MatchIterator} and \texttt{BoundedMatchIterator} are two iterator
implementations used in \texttt{MatchFinder}. \texttt{MatchIterator} simply
iterates on a match tree depth-first, either left-to-right or
right-to-left. \texttt{BoundedMatchIterator} encapsulates a
\texttt{MatchIterator} but only returns the nodes between or outside the
\emph{left-match} and \emph{right-match} (as described in section
\ref{finding_submatches}).

%%%%%%%%%%%%%%%%%%%%%%%%%%%%%%%%%%%%%%%%
\subsection{compiler}
\label{compiler_tour}

This package concerns itself with the compilation of macros and the expansion of
macro calls.

The \texttt{Macro} class holds all attributes of a macro, as described in the
user manual.

The \texttt{MacroCompiler} class is responsible for turning a macro into
compiled bytecode. It requires three key pieces of information: the macro name,
the expanded macro body, and the list of \texttt{import} statements that should
be included in the compiled source (derived from the macro file prelude).

The \texttt{compiler.java} package is used to turn the source pieced from these
pieces of information into bytecode.

A match tree is an immutable structure. The \texttt{MatchTreeTransformer}
abstract class holds the logic to apply a recursive transformation on a match
tree, and rebuild only the parts of the tree which have changed. The subtrees
that have not changed are shared between the new and the old tree. Implementing
classes just need to implement a method that transforms a single \texttt{Match}.

There are three classes implementing \texttt{MatchTreeTransformer}. The
\texttt{MacroExpander} class expands encountered macros according to their
expansion strategies, as described in section \ref{expansion_strategies} (see
also sections \ref{macro_expansion_intro} and \ref{raw_macros} on macro
expansion).

The \texttt{PostParseTransformer} and \texttt{PostExpansionTransformer} classes
from the \texttt{CallbacksTransformers.java} file apply the transformations that
result from some of the callbacks specified in \texttt{MatchCallbacks} objects.

The \texttt{PostParser} class simply applies the transformations from
\texttt{PostParseTransformer}, \texttt{MacroExpander} and
\texttt{PostExpansionTransformer}, in that order.

Finally, the \texttt{QuotationMacro} class is a bootstrapped macro that expands
the quotation syntax (see section \ref{quotation_manual}) into calls to Java
methods defined in the \texttt{compiler.util.Quoter} class. We say
\emph{bootstrapped} macro, because, while it acts exactly like a macro, it was
hand-coded and not compiled via \texttt{MacroCompiler}.

%%%%%%%%%%%%%%%%%%%%%%%%%%%%%%%%%%%%%%%%
\subsection{compiler.java}

The classes in this package interface with the \texttt{javax.tools} package from
the JDK libraries. \texttt{DynamicJavaCompiler} defines our own interface to the
compiler, which takes \texttt{javax.tools.JavaFileObject} instances and returns
a list of \texttt{CompiledClass} instances (one per class defined in the file
objects).

The \texttt{javax.tools.JavaFileObject} class represents a Java compilation
unit, meaning a source file. The \texttt{CompiledClass} class represents the
bytecode obtained by compiling a Java class.

The \texttt{StringJavaFileObject} class extends the
\texttt{javax.tools.JavaFileObject} class to allow us to use strings as the
source code of compilation units.

Finally, the \texttt{MemoryClassLoader} is a class loader that can hold the
bytecode of classes in memory. When searching for a class to load, it will first
search among the classes whose bytecode is in memory, then only search for
\texttt{.class} files on the disk.

%%%%%%%%%%%%%%%%%%%%%%%%%%%%%%%%%%%%%%%%
\subsection{compiler.util}

The facilities in \texttt{MatchCreator} are described in section
\ref{match_class}.

The \texttt{StringMatcher} class holds a small helper method that abstracts away
the boilerplate needed to match a source string to a grammar rule.

The class \texttt{PEGCompiler} takes a string representing a macro's syntactic
specification and returns the corresponding \texttt{Expression}. This is
currently unused except in tests.

Last but not least, the \texttt{Quoter} class implements the quotation
primitives. The way this class relates to quotation as described in the manual
is explained in section \ref{implem_quotations}.

%%%%%%%%%%%%%%%%%%%%%%%%%%%%%%%%%%%%%%%%
\subsection{compiler.macros}

This is the package in which compiled macros will be placed. It only contains
the \texttt{MacroInterface} interface, which is the interface that the class
generated for each macro must implement.

%%%%%%%%%%%%%%%%%%%%%%%%%%%%%%%%%%%%%%%%
\subsection{util and util.apache}

The package holds a collection of data structures and utilities that ease some
programming tasks, but are not inherently tied to our program's logic.

\texttt{util.apache} has classes extracted from the Apache Commons library. In
particular, they deal with source code escaping.

%%%%%%%%%%%%%%%%%%%%%%%%%%%%%%%%%%%%%%%%%%%%%%%%%%%%%%%%%%%%%%%%%%%%%%%%%%%%%%%%
\section{The \texttt{require} Statement}

Introduced in section \ref{require_manual} of the user manual, the
\texttt{require} statement extends the functionality of the \texttt{import}
statement by explicitly specifying which source file is being depended upon.

Central to understanding the \texttt{require} statement is the notion of
\emph{dependency}, and the difference between a run-time and compile-time
dependency. We first describe see how Java handles dependencies before
explaining how caxap deals with dependencies at compile time.

%%%%%%%%%%%%%%%%%%%%%%%%%%%%%%%%%%%%%%%%%%%%%%%%%%%%%%%%%%%%%%%%%%%%%%%%%%%%%%%%
\subsection{Dependencies and the Java Compiler}

We describe here how dependencies are handled in Java. This helps explaining why
we needed to introduce \texttt{require} statements instead of hijacking the Java
\texttt{import} statement.

We base our discussion on the Oracle Java compiler (\emph{javac}), but most
things are applicable to all Java compilers targeting the Java Virtual Machine
(JVM).

%%%%%%%%%%%%%%%%%%%%%%%%%%%%%%%%%%%%%%%%%%%%%%%%%%%%%%%%%%%%%%%%%%%%%%%%%%%%%%%%
\subsubsection{Dependencies and \texttt{import} Statements}

The first and foremost thing to understand is that, in Java, the dependencies of
a class are the set of other classes referenced in the class.

As most people will tell you, \texttt{import} statements are related to
dependencies. What most people don't realize however, is that, in Java,
\texttt{import} statement are merely a convenience to avoid using the fully
qualified name of each class (e.g. using \texttt{List} instead of
\texttt{java.util.List}). Every \texttt{import} statement in Java could be
removed at the cost of verbosity. This would not change the class dependencies.

There are two kinds of \texttt{import} statements in Java. Regular
\texttt{import} statements import a class name from another package. Static
imports - denoted by the keyword \texttt{static} - import the name of a static
method, field or class from another class. Both kind can also end with a
wildcard (\lstinline{'*'}) rather than a name. In that case all classes in the
package or static members in the class are imported.

When javac encounters an identifier that could potentially be a class name,
javac will attempt to resolve the identifier as a class in the same package as
the class being compiled. This means that all classes from the same package are
implicitly imported in every class.

%%%%%%%%%%%%%%%%%%%%%%%%%%%%%%%%%%%%%%%%%%%%%%%%%%%%%%%%%%%%%%%%%%%%%%%%%%%%%%%%
\subsubsection{Dependencies at Compile Time and Run Time}

While dependencies are checked by the compiler at compile time, the result of
compilation is a set of \texttt{.class} file containing bytecode. There is one
\texttt{.class} file per class defined in the source file(s). This means that
the JVM must resolve the dependencies at run time, much like how dynamically
linked library work. The JVM does this by looking for the appropriate
\texttt{.class} file on the classpath. The classpath is a list of paths where
the compiler looks for \texttt{.class} files. The user can add paths to the
classpath using the command line options of the JVM.

The compiler, which takes a set of source file as input, checks the dependencies
of those files at compile time. For each import statement, the compiler ensures
that the imported classes or class members are defined. The compiler looks for
those in two places. First, inside \texttt{.class} files which are found on the
classpath. The compiler also looks for definitions in the source files
themselves, since the relevant source file might not have been compiled to
\texttt{.class} files yet.

Since a single source file can contain the definition of multiple classes, there
can be more than one \texttt{.class} file emitted for each source file.

%%%%%%%%%%%%%%%%%%%%%%%%%%%%%%%%%%%%%%%%%%%%%%%%%%%%%%%%%%%%%%%%%%%%%%%%%%%%%%%%
\subsection{Problems with the \texttt{import} Statements}
\label{import_problems}

We can't use \texttt{import} statements to specify dependencies between source
files. First, the aim of \texttt{import} statement is not to specify
dependencies, but to avoid typing out the fully qualified name of each class
each time it is used. Second, \texttt{import} statements reference
classes. There is an asymmetry between classes and source files: a source file
can contain the definition of multiples classes.

The fully qualified class names used in \texttt{import} statements are also
ambiguous with regard to the location of the containing source file. Consider
the statement ``\texttt{import com.company.project.Class;}''. The class
\texttt{Class} could simply be defined in the file
\texttt{com/company/project/Class.java}. The class could also be defined in any
other file in the package, if it isn't public. Worse, it could be defined inside
the file \texttt{com/company/project.java}, if \texttt{Class} is a nested class
of the class \texttt{project}. While convention dictates that class names begin
with an uppercase letter, this is not enforced by the Java compiler.

Even if we were willing to find the file in which a class was defined, it would
prove to be impossible in the presence of macros. Indeed, a macro call could
``hide'' the class definition, and we cannot expand macros before resolving
dependencies.

Clearly, a new syntax was required to deal with source file dependencies.

%%%%%%%%%%%%%%%%%%%%%%%%%%%%%%%%%%%%%%%%%%%%%%%%%%%%%%%%%%%%%%%%%%%%%%%%%%%%%%%%
\subsection{Why do we need compile-time dependencies?}
\label{why_source_deps}

As explained in section \ref{require_manual}, we must know of dependencies
between source files so that we can find a linear ordering of source files. We
then compile the files in that order.

If a source file uses a macro, then the macro's expansion code needs to be
compiled before the source file can be compiled. This automatically precludes
cyclic dependencies between macros: When trying to compile a macro's expansion
code, we would need the compiled expansion code of another macro, the
compilation of which would need the compiled expansion code of the macro we were
trying to compile in the first place.

But wouldn't it still be possible to have cyclic dependencies between macro
files, as long as there are no cyclic dependencies between macros? The problem
is that since a macro body can contain macro calls, we can only know where the
macro body ends after having seen the syntactic specification of the macros it
calls. But we can't know in advance which macros the body calls; hence we need
to see all required macros syntactic specification, and thus parse all required
macros' bodies.

There are a few ways around this. We could specify the macro depended upon at
the macro definition level. We could move all syntactic specifications to the
top of the file. Or we could reserve a token to indicate the end of a macro's
body, which could not be used in user-defined syntax. All those fixes seem to be
more bothersome than they are worth.

If a macro file uses a class in its expansion code, then this class needs to be
located by the Java compiler. This means that the compiler must have access to a
\texttt{.class} for the macro, or to a source file containing its definition. In
the second case, we still need to fully pre-process the defining source file;
hence there is still a dependency link.

Dependencies between two regular source files need to be specified for two
reasons. The first is that they can lead to a transitive dependency on a
macro. Let A, B be regular source files, and C a macro file. If a A depends on B
and B depends on C, then C must be compiled before A can be compiled.

Second, caxap compiles files only if they are somehow required for macro
expansion. If we did not specify a dependency on another source file, we would
not know we need to compile it.

Currently, files are compiled one at a time. This forbids some well-behaved
cases such as the following example. Let A, B be regular source files containing
no macro calls, and C a macro file. C depends on A, A depends on B and B depends
on A. There is a cyclic dependency between A and B that will be rejected by
caxap. Yet, this dependency could be encoded using \texttt{import} statement and
would compile just fine if we supplied both A and B to the Java compiler.

In the future, we might try to produce some kind partial ordering that would
allow us to compile multiple files together, instead of a total ordering.

%%%%%%%%%%%%%%%%%%%%%%%%%%%%%%%%%%%%%%%%%%%%%%%%%%%%%%%%%%%%%%%%%%%%%%%%%%%%%%%%
\section{Finding Submatches}

The design of the match finding API described in section \ref{match_api} is
pretty interesting. I wanted to make an API that would make it easy to retrieve
submatches in a match tree, without needing to refer back to the grammar
constantly.

The goal of the API is to be able to express the position of a match in the
exact same way as people think about it. It turns out that if you want to
describe the location of a match, you'll probably anchor your description to
some element that precedes or follows that match. You'll use words such as
\emph{after}, \emph{before} or \emph{between}. If multiple matches satisfy the
requirements, these prepositions will typically be complemented by words such as
\emph{first} or \emph{last}.

This explains the overall structure of the API, particularly its
\emph{first}/\emph{after}/\emph{all} and
\emph{after}/\emph{before}/\emph{between}/\emph{outside} distinctions.

Moreover, the variety of available \texttt{MachSpec} (see section
\ref{matchspec_manual}) allows specifying matches even if the exact name of the
grammar rule is not known. You can express conjunction and disjunction of specs
with \texttt{AndSpec} and or \texttt{OrSpec}. And of course, if something is
missing, you can always roll your own \texttt{MatchSpec} implementation.

In our experience - which is at present somewhat limited - the system succeeds
at making most request terse and intuitive. The approach does have a drawback in
that it is not overly efficient, since it is basically a tree search.

The idea of finding data in an annotated tree is not a new idea. In fact, it is
rather pervasive: the cascading stylesheet language (CSS) \cite{css_page} used
on almost every website is constructed around this idea. CSS tries to match HTML
tags to apply styling on them. CSS uses the term \emph{selector} instead of
\emph{specification}.

There are differences however. First, CSS is skewed toward matching potentially
large set of scattered items, whereas we tend to more stringent in our
requirements. Unique HTML tags in a page layout are given a unique \emph{id},
which further simplifies CSS' task. Another difference is that CSS match items
recursively, which makes sense in its context.

There are other query languages operating on trees that work somewhat
similarly. One example is the XPath query language \cite{xpath}, which is part
of the XSLT (Extensible Stylesheet Language Transformations) language.

%%%%%%%%%%%%%%%%%%%%%%%%%%%%%%%%%%%%%%%%%%%%%%%%%%%%%%%%%%%%%%%%%%%%%%%%%%%%%%%%
\section{Quotations}
\label{implem_quotations}

The most difficult thing about quotations is understanding how they work. After
that, implementing them is not all that hard, if you get the right ideas.

As mentioned in section \ref{compiler_tour}, the quotation syntax acts like a
macro, implemented in \texttt{compiler.QuotationMacro}.

Simple quotations expand to the \texttt{compiler.util.Quoter.primitiveQuote()}
method. For instance, \lstinline{'expression[ 42 ]'} expands to
\texttt{primitiveQuote("expression", "42")}. The first parameter is a rule name,
and the second is the source fragment, escaped so as to form a valid Java
string.

\texttt{primitiveQuote()}, which is overloaded, can also take a list of
\texttt{MatchSpec} objects as a list or as a variadic parameter. The method
checks if the match resulting from the quotation satisfies all passed
specs. This parameter is not used for simple quotations.

Quasiquotations expand to the \texttt{compiler.util.Quoter.dynamicQuote()}
method. For instance, \lstinline{`expression[ 1337 + #method() ]`} expands to
\lstinline{dynamicQuote("expression", "1337 + #1", method())}. The reformulation
is necessary in order to evaluate unquoted expression in Java. The
\texttt{dynamicQuote()} method inserts the result of all unquoted expressions
into the source fragment, before passing it to \texttt{primitiveQuote()}. If the
unquoted expression is a \texttt{Match}, \texttt{dynamicQuote()} also constructs
a special \texttt{MatchSpec} instance that ensures that the match appears as a
submatch in the quotation's result, and passes this spec to
\texttt{primitiveQuote()}.

%%%%%%%%%%%%%%%%%%%%%%%%%%%%%%%%%%%%%%%%%%%%%%%%%%%%%%%%%%%%%%%%%%%%%%%%%%%%%%%%
\section{The Java Grammar}
\label{our_java_grammar}

The PEG grammar used to parse the Java language is entirely hand crafted. There
are a few PEG grammars for Java to be found online, namely as part of the
Parboiled \cite{parboiled}, Mouse \cite{redziejowski2009} and Rats!
\cite{grimm2006} PEG parsers. None truly satisfied me: The Parboiled grammar was
for Java 6; the Rats! grammar used a copious amount of Rats! specific extensions
and both the Mouse and Parboiled grammars was based upon the CFG grammar from
the Java Language Specification \cite{jls} which is explicitly designed to
minimize lookahead. I also found a few errors in the Mouse grammar, which is the
only grammar I thoroughly scrutinized.

My idea for the perfect Java grammar was that it should reflect how one thinks
about the language. In other words, the grammar should map cleanly to our
intuitions about the language. It should also be highly granular: since each
rule is a potential macro insertion point, there should be enough rules to
capture all the places where a user might want to insert a macro, and the rules
should represent meaningful Java entities. There should not be too large a
divide between the syntactic and semantic view of the language.

Unfortunately, grammars designed to minimize lookahead violate those principles,
notably by \emph{factoring out} common prefixes of rules. This means that we are
left with very grave gaps: there are for instance no rule that designate array
access or method invocation directly. Instead, we have a rule with an identifier
as prefix, which is followed by any number of suffixes such as array access or
method invocations. In the context of our macro system, that does constitute a
problem.

Yet, my grammar is not devoid of problems. First, it ended up picking some of
the gaps highlighted above because the PEG formalism does not support left
recursion. While we can emulate left-recursion using the ``prefix followed by a
repetition of suffixes'' pattern, left-recursion produces a nesting of rules
that properly associates each array access / method invocation with its prefix
operand. As explained in section \ref{peg_left_recursion}, there are ways to
introduce left recursion, but I chose to move on with the implementation
instead.

The example from \texttt{examples/src/pkg/ArraySlice.javam} shows the hoops one
has to jump through in the absence of left-recursion. It's not quite
insurmountable (the whole macro definition is less than 30 lines), but it
requires some careful thinking.

Second, it made my grammar too specific. Each programming languages has
constraints on valid programs that cannot be expressed by a grammar, such as
typing constraints. There are other constraints which can be expressed by
grammars, but are best left out. Consider for instance modifier keywords such as
\texttt{private}, \texttt{abstract}, etc. Not all keywords are allowed for all
constructs, the keywords can appear in any order, but each keyword cannot appear
more than once per construct. This is a distinctively non-trivial set of
constraints to embed in a grammar, yet I tried nonetheless. It was a bad idea,
as that made the grammar much more complicated than it should have been. On the
bright side, my grammar is the most precise computer-readable specification of
Java 7 syntax that I am aware of.

The grammar probably still contains a few problems I failed to discover.
However, the fact that we succeed in parsing the whole OpenJDK source tree at
least gives us some confidence in the grammar's correctness (see section
\ref{testing} for details).

Because the source also includes test files containing incorrect syntax on
purpose, I had to manually exclude about 150 source files from the test. I only
checked about 15 of them manually to see if they indeed contained syntax error,
which was the case. All those files appear in test folders, and many of them
have names that indicate quite explicitly they shouldn't parse. The excluded
files are listed in the file \texttt{src/test/main/OpenJDKExcludes.java}.

Finally, it is quite clear that the design of the grammar impacts performance
negatively. First, because the grammar tries to map to semantic constructs and
does not factor out common prefixes, it makes memoization absolutely
essential. Without it, exponential behavior kicks in and yields absolutely
preposterous parse times. Contrast this with the Mouse \cite{redziejowski2009}
parser generator, whose author says performs best with a very limited amount of
memoization. With Mouse, doing no memoization at all consistently outperforms
full memoization. We expand on our performance problems in section \ref{perf}.
