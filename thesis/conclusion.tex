\chapter{Conclusion}

%%%%%%%%%%%%%%%%%%%%%%%%%%%%%%%%%%%%%%%%%%%%%%%%%%%%%%%%%%%%%%%%%%%%%%%%%%%%%%%%
\subsubsection{What Was Done}

I have implemented a macro framework on top of the Java programming
language. This framework acts as a pre-processor for Java sources. It relies on
parsing expression grammars to specify the syntax of macros. Its prime design
objective was generality: I didn't want to restrict the syntax the user could
introduce into the language, nor how he could process this syntax. caxap's
macros are therefore procedural, meaning arbitrary code is run at compile time
to generate a macro's expansion.

I hope to have shown that such an approach was practical. The framework can now
be used to implement macros of practical values, and unlock a whole range of
abstractions that were previously impossible. There are problems, which we talk
about in the next section. But none of these problems form a major road block. I
am confident in my ability to reduce the impact of those problems significantly,
or even eliminate them totally, in the future.

One of the assumptions I made was that by designing a very general core, the
framework could be extended with facilities that would make working with macros
easier in common cases. I think that our implementation of quasiquotation
(section \ref{quotation_manual}) and match finders (section \ref{match_api})
demonstrate this assumption to be true.

%%%%%%%%%%%%%%%%%%%%%%%%%%%%%%%%%%%%%%%%%%%%%%%%%%%%%%%%%%%%%%%%%%%%%%%%%%%%%%%%
\subsubsection{The Way Forward}

The work on caxap is not complete. Its main issue is a poor user
experience. This manifests itself primarily in three aspects: poor performance,
confusing error reports and lack of IDE integration. We explore potential
solutions to those issues in section \ref{future_work}.

While fixing the user experience will be my top priority, there are a few
features I would like to see in caxap. Chief amongst them is the ability to
perform non-local transformations, as discussed in section
\ref{global_transformations}.

Finally, I would like to make the system capable of extending itself. Meaning
that enough functionality would be exposed to allow convenience utilities such
as quasiquotation and match finding to be implemented by users of caxap without
needing to dig in its internals. I do believe this objective is not that far out
of reach.
